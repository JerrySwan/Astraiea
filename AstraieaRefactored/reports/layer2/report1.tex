\documentclass[]{article}
\usepackage[usenames,dvipsnames]{color}
\usepackage{longtable}
\begin{document}

\section{Experimental Results (Tue Nov 24 23:28:42 GMT 2015)}
Experiments were performed using the \textsc{Astraiea}\footnote{This file has been automatically generated by Astraiea} statistical testing framework \cite{Neumann:2014:EET:2598394.2609850},
which performs tests in accordance with the guidelines of ``A Hitchhikers guide to statistical testing''
by Briand and Arcuri~\cite{Arcuri2012}.
Unless a different reference is given, the tests are performed as described in this paper.

Statistical testing was carried out as follows: 
This data was obtained from runs on multiple artefacts. Each artefact is treated as a single run for the purpose of statistical comparison and so, the experimental description below,  the number of runs refers to the number of artefacts. For each artefact 1 repeated tests were carried out and the median of these results was used as the result for that artefact for the purposes of subsequent statistical testing.\begin{itemize}
\item{All experiments were repeated 30 times ($n$=30)}
\item{The Wilcoxon/ Mann Whitney U Significance Test was used.}
\item{The Vargha Delaney Effect Size Test was used. Effect size testing is essential in addition to significance testing as it demonstrates the magnitude of the difference between two samples. With a large enough number of experiments (large enough $n$), the results of two different generating techniques are likely to be different to a statistically significant extent. Effect size testing is needed to show that this difference is useful.}
\end{itemize}A complete list of p value tests carried out is shown in table~\ref{p value tests}, with $n$ corresponding to the number of samples in each data set.
\begin{center}
\begin{longtable}{|l|l|l|}
\caption[P Value Tests]{P Value Tests} \label{p value tests} \\ 
\hline \multicolumn{1}{|c|}{\textbf{n}} &  \multicolumn{1}{|c|}{\textbf{Notes}} &  \multicolumn{1}{|c|}{\textbf{P-Value}}
\\ \hline 
\endfirsthead 
\multicolumn{3}{c}{{\bfseries \tablename\ \thetable{} -- continued from previous page}} \\ 
 \hline 
 \multicolumn{1}{|c|}{\textbf{n}} &  \multicolumn{1}{|c|}{\textbf{Notes}} &  \multicolumn{1}{|c|}{\textbf{P-Value}}
\endhead 
\hline \multicolumn{3}{|r|}{{Continued on next page}} \\ \hline 
\endfoot 
\hline 
\endlastfoot 

\hline
\end{longtable}
\end{center}

The final test, and the test on which effect size was calculated, was carried out using an $n$ of 30. 

The final results from comparing null and null are as follows:
\begin{itemize}
\item{difference is significant (p value of 0.02558535265900963, threshold of 0.05)}
\item{the effect size is 0.3322222222222222, confidence intervals (obtained by bootstrapping)= 0.18894444444444444 - 0.4922222222222222}
\item{null is lower than null}
\end{itemize}Statistical testing was carried out as follows: 
This data was obtained from runs on multiple artefacts. Each artefact is treated as a single run for the purpose of statistical comparison and so, the experimental description below,  the number of runs refers to the number of artefacts. For each artefact 1 repeated tests were carried out and the median of these results was used as the result for that artefact for the purposes of subsequent statistical testing.\begin{itemize}
\item{All experiments were repeated 30 times ($n$=30)}
\item{The Brunner Munzel P Value Significance Test was used. Brunner Munzel is used in place of the  Wilcoxon/ Mann Whitney U test recommended in Arcuri's paper~\cite{Arcuri2012} because the Brunner Munzel test is tolerant of heteroscedastic data whereas the Wilcoxon/ Mann Whitney U test is not~\cite{Brunner2000}.}
\item{The Vargha Delaney Effect Size Test was used. Effect size testing is essential in addition to significance testing as it demonstrates the magnitude of the difference between two samples. With a large enough number of experiments (large enough $n$), the results of two different generating techniques are likely to be different to a statistically significant extent. Effect size testing is needed to show that this difference is useful.}
\end{itemize}A complete list of p value tests carried out is shown in table~\ref{p value tests}, with $n$ corresponding to the number of samples in each data set.
\begin{center}
\begin{longtable}{|l|l|l|}
\caption[P Value Tests]{P Value Tests} \label{p value tests} \\ 
\hline \multicolumn{1}{|c|}{\textbf{n}} &  \multicolumn{1}{|c|}{\textbf{Notes}} &  \multicolumn{1}{|c|}{\textbf{P-Value}}
\\ \hline 
\endfirsthead 
\multicolumn{3}{c}{{\bfseries \tablename\ \thetable{} -- continued from previous page}} \\ 
 \hline 
 \multicolumn{1}{|c|}{\textbf{n}} &  \multicolumn{1}{|c|}{\textbf{Notes}} &  \multicolumn{1}{|c|}{\textbf{P-Value}}
\endhead 
\hline \multicolumn{3}{|r|}{{Continued on next page}} \\ \hline 
\endfoot 
\hline 
\endlastfoot 

\hline
\end{longtable}
\end{center}

The final test, and the test on which effect size was calculated, was carried out using an $n$ of 30. 

The final results from comparing null and null are as follows:
\begin{itemize}
\item{difference is significant (p value of 2.0199143460808422E-4, threshold of 0.05)}
\item{the effect size is 0.24888888888888888, confidence intervals (obtained by bootstrapping)= 0.13555555555555557 - 0.37444444444444447}
\item{null is lower than null}
\end{itemize}Statistical testing was carried out as follows: 
This data was obtained from runs on multiple artefacts. Each artefact is treated as a single run for the purpose of statistical comparison and so, the experimental description below,  the number of runs refers to the number of artefacts. For each artefact 1 repeated tests were carried out and the median of these results was used as the result for that artefact for the purposes of subsequent statistical testing.\begin{itemize}
\item{All experiments were repeated 30 times ($n$=30)}
\item{The data is paired. A paired version of the Wilcoxon/Mann-Whitney U test for significance will be used.}
\item{The Vargha Delaney Effect Size Test was used. Effect size testing is essential in addition to significance testing as it demonstrates the magnitude of the difference between two samples. With a large enough number of experiments (large enough $n$), the results of two different generating techniques are likely to be different to a statistically significant extent. Effect size testing is needed to show that this difference is useful.}
\end{itemize}A complete list of p value tests carried out is shown in table~\ref{p value tests}, with $n$ corresponding to the number of samples in each data set.
\begin{center}
\begin{longtable}{|l|l|l|}
\caption[P Value Tests]{P Value Tests} \label{p value tests} \\ 
\hline \multicolumn{1}{|c|}{\textbf{n}} &  \multicolumn{1}{|c|}{\textbf{Notes}} &  \multicolumn{1}{|c|}{\textbf{P-Value}}
\\ \hline 
\endfirsthead 
\multicolumn{3}{c}{{\bfseries \tablename\ \thetable{} -- continued from previous page}} \\ 
 \hline 
 \multicolumn{1}{|c|}{\textbf{n}} &  \multicolumn{1}{|c|}{\textbf{Notes}} &  \multicolumn{1}{|c|}{\textbf{P-Value}}
\endhead 
\hline \multicolumn{3}{|r|}{{Continued on next page}} \\ \hline 
\endfoot 
\hline 
\endlastfoot 

\hline
\end{longtable}
\end{center}

The final test, and the test on which effect size was calculated, was carried out using an $n$ of 30. 

The final results from comparing null and null are as follows:
\begin{itemize}
\item{difference is significant (p value of 2.6098434503518675E-6, threshold of 0.05)}
\item{the effect size is 2.5}
\item{null is greater than null}
\end{itemize}Statistical testing was carried out as follows: 
This data was obtained from runs on multiple artefacts. Each artefact is treated as a single run for the purpose of statistical comparison and so, the experimental description below,  the number of runs refers to the number of artefacts. For each artefact 1 repeated tests were carried out and the median of these results was used as the result for that artefact for the purposes of subsequent statistical testing.\begin{itemize}
\item{All experiments were repeated 30 times ($n$=30)}
\item{The data is dichotomous and paired and so the Fisher test of statistical significance was used followed by the Odds Ratio test of effect size.}
\end{itemize}A complete list of p value tests carried out is shown in table~\ref{p value tests}, with $n$ corresponding to the number of samples in each data set.
\begin{center}
\begin{longtable}{|l|l|l|}
\caption[P Value Tests]{P Value Tests} \label{p value tests} \\ 
\hline \multicolumn{1}{|c|}{\textbf{n}} &  \multicolumn{1}{|c|}{\textbf{Notes}} &  \multicolumn{1}{|c|}{\textbf{P-Value}}
\\ \hline 
\endfirsthead 
\multicolumn{3}{c}{{\bfseries \tablename\ \thetable{} -- continued from previous page}} \\ 
 \hline 
 \multicolumn{1}{|c|}{\textbf{n}} &  \multicolumn{1}{|c|}{\textbf{Notes}} &  \multicolumn{1}{|c|}{\textbf{P-Value}}
\endhead 
\hline \multicolumn{3}{|r|}{{Continued on next page}} \\ \hline 
\endfoot 
\hline 
\endlastfoot 
