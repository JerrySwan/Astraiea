\documentclass[]{article}
\usepackage[usenames,dvipsnames]{color}
\usepackage{longtable}
\begin{document}

\section{Experimental Results (Wed Nov 25 16:41:21 GMT 2015)}
Experiments were performed using the \textsc{Astraiea}\footnote{This file has been automatically generated by Astraiea} statistical testing framework \cite{Neumann:2014:EET:2598394.2609850},
which performs tests in accordance with the guidelines of ``A Hitchhikers guide to statistical testing''
by Briand and Arcuri~\cite{Arcuri2012}.
Unless a different reference is given, the tests are performed as described in this paper.

Statistical testing was carried out as follows: 
This data was obtained from runs on multiple artefacts. Each artefact is treated as a single run for the purpose of statistical comparison and so, the experimental description below,  the number of runs refers to the number of artefacts. For each artefact 1 repeated tests were carried out and the median of these results was used as the result for that artefact for the purposes of subsequent statistical testing.\begin{itemize}
\item{All experiments were repeated 30 times ($n$=30)}
\item{The data is dichotomous and paired and so the Fisher test of statistical significance was used followed by the Odds Ratio test of effect size.These tests compare boolean values, denoting a pass or a fail, from the final point in the result time series. }
\end{itemize}A complete list of p value tests carried out is shown in table~\ref{p value tests}, with $n$ corresponding to the number of samples in each data set.
\begin{center}
\begin{longtable}{|l|l|l|}
\caption[P Value Tests]{P Value Tests} \label{p value tests} \\ 
\hline \multicolumn{1}{|c|}{\textbf{n}} &  \multicolumn{1}{|c|}{\textbf{Notes}} &  \multicolumn{1}{|c|}{\textbf{P-Value}}
\\ \hline 
\endfirsthead 
\multicolumn{3}{c}{{\bfseries \tablename\ \thetable{} -- continued from previous page}} \\ 
 \hline 
 \multicolumn{1}{|c|}{\textbf{n}} &  \multicolumn{1}{|c|}{\textbf{Notes}} &  \multicolumn{1}{|c|}{\textbf{P-Value}}
\endhead 
\hline \multicolumn{3}{|r|}{{Continued on next page}} \\ \hline 
\endfoot 
\hline 
\endlastfoot 
