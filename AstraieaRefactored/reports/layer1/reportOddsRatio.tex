\documentclass[]{article}
\usepackage[usenames,dvipsnames]{color}
\usepackage{longtable}
\begin{document}

\section{Experimental Results (Sun Nov 22 09:39:28 GMT 2015)}
Experiments were performed using the \textsc{Astraiea}\footnote{This file has been automatically generated by Astraiea} statistical testing framework \cite{Neumann:2014:EET:2598394.2609850},
which performs tests in accordance with the guidelines of ``A Hitchhikers guide to statistical testing''
by Briand and Arcuri~\cite{Arcuri2012}.
Unless a different reference is given, the tests are performed as described in this paper.

Statistical testing was carried out as follows: \begin{itemize}
\item{The data is dichotomous and unpaired and so the Fisher test of statistical significance and the Odds Ratio test of effect size are used.}
\item{These results were obtained with 20 samples in each dataset}
\end{itemize}Results:
\begin{itemize}
\item{difference is significant (p value of 3.59967366786534E-4, threshold of 0.05)}
\item{the effect size is 13.444444444444445}
\item{dataA is greater than dataB}
\end{itemize}Statistical testing was carried out as follows: \begin{itemize}
\item{The data is dichotomous and unpaired and so the Fisher test of statistical significance and the Odds Ratio test of effect size are used.}
\item{These results were obtained with 20 samples in each dataset}
\end{itemize}Results:
\begin{itemize}
\item{data is not significant (p value of 0.9999999999999897, threshold of 0.05)}
\end{itemize}Statistical testing was carried out as follows: \begin{itemize}
\item{The data is dichotomous and unpaired and so the Fisher test of statistical significance and the Odds Ratio test of effect size are used.}
\item{These results were obtained with 20 samples in each dataset}
\end{itemize}Results:
\begin{itemize}
\item{difference is significant (p value of 3.59967366786534E-4, threshold of 0.05)}
\item{the effect size is 0.0743801652892562}
\item{dataA is lower than dataB}
\end{itemize}Statistical testing was carried out as follows: \begin{itemize}
\item{The data is dichotomous and unpaired and so the Fisher test of statistical significance and the Odds Ratio test of effect size are used.}
\item{These results were obtained with 20 samples in each dataset}
\end{itemize}Results:
\begin{itemize}
\item{difference is significant (p value of 4.3591979075849633E-4, threshold of 0.05)}
\item{the effect size is 41.0}
\item{dataA is greater than dataB}
\end{itemize}Statistical testing was carried out as follows: \begin{itemize}
\item{The data is dichotomous and unpaired and so the Fisher test of statistical significance and the Odds Ratio test of effect size are used.}
\item{These results were obtained with 20 samples in each dataset}
\end{itemize}Results:
\begin{itemize}
\item{difference is significant (p value of 4.3591979075849633E-4, threshold of 0.05)}
\item{the effect size is 0.024390243902439025}
\item{dataA is lower than dataB}
\end{itemize}Statistical testing was carried out as follows: \begin{itemize}
\item{The data is dichotomous and unpaired and so the Fisher test of statistical significance and the Odds Ratio test of effect size are used.}
\item{These results were obtained with 20 samples in each dataset}
\end{itemize}Results:
\begin{itemize}
\item{difference is significant (p value of 4.3591979075849324E-4, threshold of 0.05)}
\item{the effect size is 41.0}
\item{dataA is greater than dataB}
\end{itemize}Statistical testing was carried out as follows: \begin{itemize}
\item{The data is dichotomous and unpaired and so the Fisher test of statistical significance and the Odds Ratio test of effect size are used.}
\item{These results were obtained with 20 samples in each dataset}
\end{itemize}Results:
\begin{itemize}
\item{data is not significant (p value of 0.7511863074565451, threshold of 0.05)}
\end{itemize}Statistical testing was carried out as follows: \begin{itemize}
\item{The data is dichotomous and unpaired and so the Fisher test of statistical significance and the Odds Ratio test of effect size are used.}
\item{These results were obtained with 50 samples in each dataset}
\end{itemize}Results:
\begin{itemize}
\item{data is not significant (p value of 0.9999999999999819, threshold of 0.05)}
\end{itemize}Statistical testing was carried out as follows: \begin{itemize}
\item{The data is dichotomous and unpaired and so the Fisher test of statistical significance and the Odds Ratio test of effect size are used.}
\item{These results were obtained with 50 samples in each dataset}
\end{itemize}Results:
\begin{itemize}
\item{difference is significant (p value of 0.009488575318860788, threshold of 0.05)}
\item{the effect size is 0.22164412070759626}
\item{dataA is lower than dataB}
\end{itemize}Statistical testing was carried out as follows: \begin{itemize}
\item{The data is dichotomous and unpaired and so the Fisher test of statistical significance and the Odds Ratio test of effect size are used.}
\item{These results were obtained with 50 samples in each dataset}
\end{itemize}Results:
\begin{itemize}
\item{difference is significant (p value of 0.008471290461855314, threshold of 0.05)}
\item{the effect size is 0.3182143709715922}
\item{dataA is lower than dataB}
\end{itemize}Statistical testing was carried out as follows: \begin{itemize}
\item{The data is dichotomous and unpaired and so the Fisher test of statistical significance and the Odds Ratio test of effect size are used.}
\item{These results were obtained with 50 samples in each dataset}
\end{itemize}Results:
\begin{itemize}
\item{difference is significant (p value of 5.487013885134591E-10, threshold of 0.05)}
\item{the effect size is 16.849624060150376}
\item{dataA is greater than dataB}
\end{itemize}Statistical testing was carried out as follows: \begin{itemize}
\item{The data is dichotomous and unpaired and so the Fisher test of statistical significance and the Odds Ratio test of effect size are used.}
\item{These results were obtained with 20 runs of metaheuristic dataA and 50 runs of metaheuristic dataB}
\end{itemize}Results:
\begin{itemize}
\item{difference is significant (p value of 4.719176169198642E-6, threshold of 0.05)}
\item{the effect size is 0.0707070707070707}
\item{dataA is lower than dataB}
\end{itemize}