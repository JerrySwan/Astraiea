\documentclass[]{article}
\usepackage[usenames,dvipsnames]{color}
\usepackage{longtable}
\begin{document}

\section{Experimental Results (Wed Feb 17 08:55:57 GMT 2016)}
Experiments were performed using the \textsc{Astraiea}\footnote{This file has been automatically generated by Astraiea} statistical testing framework \cite{Neumann:2014:EET:2598394.2609850},
which performs tests in accordance with the guidelines of ``A Hitchhikers guide to statistical testing''
by Briand and Arcuri~\cite{Arcuri2012}.
Unless a different reference is given, the tests are performed as described in this paper.

Statistical testing was carried out as follows: \begin{itemize}
\item{The data is dichotomous and paired and so the McNemar test of statistical significance~\cite{Gibbons2011} and the Matched Odds Ratio test of effect size are used.}
\item{These results were obtained with 20 samples in each dataset}
\end{itemize}Results:
\begin{itemize}
\item{difference is significant (p value of 0.005959526471941867, threshold of 0.05)}
\item{the effect size is 5.8}
\item{dataA is greater than dataB}
\end{itemize}Statistical testing was carried out as follows: \begin{itemize}
\item{The data is dichotomous and paired and so the McNemar test of statistical significance~\cite{Gibbons2011} and the Matched Odds Ratio test of effect size are used.}
\item{These results were obtained with 20 samples in each dataset}
\end{itemize}Results:
\begin{itemize}
\item{difference is significant (p value of 0.009521891187027176, threshold of 0.05)}
\item{the effect size is 0.22580645161290322}
\item{dataA is lower than dataB}
\end{itemize}Statistical testing was carried out as follows: \begin{itemize}
\item{The data is dichotomous and paired and so the McNemar test of statistical significance~\cite{Gibbons2011} and the Matched Odds Ratio test of effect size are used.}
\item{These results were obtained with 20 samples in each dataset}
\end{itemize}Results:
\begin{itemize}
\item{difference is significant (p value of 0.004426525859281161, threshold of 0.05)}
\item{the effect size is 21.0}
\item{dataA is greater than dataB}
\end{itemize}Statistical testing was carried out as follows: \begin{itemize}
\item{The data is dichotomous and paired and so the McNemar test of statistical significance~\cite{Gibbons2011} and the Matched Odds Ratio test of effect size are used.}
\item{These results were obtained with 20 samples in each dataset}
\end{itemize}Results:
\begin{itemize}
\item{difference is significant (p value of 0.004426525859281161, threshold of 0.05)}
\item{the effect size is 0.047619047619047616}
\item{dataA is lower than dataB}
\end{itemize}Statistical testing was carried out as follows: \begin{itemize}
\item{The data is dichotomous and paired and so the McNemar test of statistical significance~\cite{Gibbons2011} and the Matched Odds Ratio test of effect size are used.}
\item{These results were obtained with 20 samples in each dataset}
\end{itemize}Results:
\begin{itemize}
\item{difference is significant (p value of 0.004426525859281161, threshold of 0.05)}
\item{the effect size is 21.0}
\item{dataA is greater than dataB}
\end{itemize}Statistical testing was carried out as follows: \begin{itemize}
\item{The data is dichotomous and paired and so the McNemar test of statistical significance~\cite{Gibbons2011} and the Matched Odds Ratio test of effect size are used.}
\item{These results were obtained with 20 samples in each dataset}
\end{itemize}Results:
\begin{itemize}
\item{data is not significant (p value of 0.751829634040615, threshold of 0.05)}
\end{itemize}Statistical testing was carried out as follows: \begin{itemize}
\item{The data is dichotomous and paired and so the McNemar test of statistical significance~\cite{Gibbons2011} and the Matched Odds Ratio test of effect size are used.}
\item{These results were obtained with 50 samples in each dataset}
\end{itemize}Results:
\begin{itemize}
\item{difference is significant (p value of 5.202443477502205E-4, threshold of 0.05)}
\item{the effect size is 0.16279069767441862}
\item{dataA is lower than dataB}
\end{itemize}Statistical testing was carried out as follows: \begin{itemize}
\item{The data is dichotomous and paired and so the McNemar test of statistical significance~\cite{Gibbons2011} and the Matched Odds Ratio test of effect size are used.}
\item{These results were obtained with 50 samples in each dataset}
\end{itemize}Results:
\begin{itemize}
\item{data is not significant (p value of 0.40424849472690594, threshold of 0.05)}
\end{itemize}Statistical testing was carried out as follows: \begin{itemize}
\item{The data is dichotomous and paired and so the McNemar test of statistical significance~\cite{Gibbons2011} and the Matched Odds Ratio test of effect size are used.}
\item{These results were obtained with 50 samples in each dataset}
\end{itemize}Results:
\begin{itemize}
\item{data is not significant (p value of 0.37675911779843696, threshold of 0.05)}
\end{itemize}Statistical testing was carried out as follows: \begin{itemize}
\item{The data is dichotomous and paired and so the McNemar test of statistical significance~\cite{Gibbons2011} and the Matched Odds Ratio test of effect size are used.}
\item{These results were obtained with 50 samples in each dataset}
\end{itemize}Results:
\begin{itemize}
\item{data is not significant (p value of 0.08085559839491796, threshold of 0.05)}
\end{itemize}Statistical testing was carried out as follows: \begin{itemize}
\item{The data is dichotomous and paired and so the McNemar test of statistical significance~\cite{Gibbons2011} and the Matched Odds Ratio test of effect size are used.}
\item{These results were obtained with 100 samples in each dataset}
\end{itemize}Results:
\begin{itemize}
\item{difference is significant (p value of 5.225753952849965E-4, threshold of 0.05)}
\item{the effect size is 4.636363636363637}
\item{dataA is greater than dataB}
\end{itemize}Statistical testing was carried out as follows: \begin{itemize}
\item{The data is dichotomous and paired and so the McNemar test of statistical significance~\cite{Gibbons2011} and the Matched Odds Ratio test of effect size are used.}
\item{These results were obtained with 200 samples in each dataset}
\end{itemize}Results:
\begin{itemize}
\item{data is not significant (p value of 1.0, threshold of 0.05)}
\end{itemize}Statistical testing was carried out as follows: \begin{itemize}
\item{The data is dichotomous and unpaired and so the Fisher test of statistical significance and the Odds Ratio test of effect size are used.}
\item{These results were obtained with 20 samples in each dataset}
\end{itemize}Results:
\begin{itemize}
\item{difference is significant (p value of 3.59967366786534E-4, threshold of 0.05)}
\item{the effect size is 13.444444444444445}
\item{dataA is greater than dataB}
\end{itemize}Statistical testing was carried out as follows: \begin{itemize}
\item{The data is dichotomous and unpaired and so the Fisher test of statistical significance and the Odds Ratio test of effect size are used.}
\item{These results were obtained with 20 samples in each dataset}
\end{itemize}Results:
\begin{itemize}
\item{data is not significant (p value of 0.9999999999999897, threshold of 0.05)}
\end{itemize}Statistical testing was carried out as follows: \begin{itemize}
\item{The data is dichotomous and unpaired and so the Fisher test of statistical significance and the Odds Ratio test of effect size are used.}
\item{These results were obtained with 20 samples in each dataset}
\end{itemize}Results:
\begin{itemize}
\item{difference is significant (p value of 3.59967366786534E-4, threshold of 0.05)}
\item{the effect size is 0.0743801652892562}
\item{dataA is lower than dataB}
\end{itemize}Statistical testing was carried out as follows: \begin{itemize}
\item{The data is dichotomous and unpaired and so the Fisher test of statistical significance and the Odds Ratio test of effect size are used.}
\item{These results were obtained with 20 samples in each dataset}
\end{itemize}Results:
\begin{itemize}
\item{difference is significant (p value of 4.3591979075849633E-4, threshold of 0.05)}
\item{the effect size is 41.0}
\item{dataA is greater than dataB}
\end{itemize}Statistical testing was carried out as follows: \begin{itemize}
\item{The data is dichotomous and unpaired and so the Fisher test of statistical significance and the Odds Ratio test of effect size are used.}
\item{These results were obtained with 20 samples in each dataset}
\end{itemize}Results:
\begin{itemize}
\item{difference is significant (p value of 4.3591979075849633E-4, threshold of 0.05)}
\item{the effect size is 0.024390243902439025}
\item{dataA is lower than dataB}
\end{itemize}Statistical testing was carried out as follows: \begin{itemize}
\item{The data is dichotomous and unpaired and so the Fisher test of statistical significance and the Odds Ratio test of effect size are used.}
\item{These results were obtained with 20 samples in each dataset}
\end{itemize}Results:
\begin{itemize}
\item{difference is significant (p value of 4.3591979075849324E-4, threshold of 0.05)}
\item{the effect size is 41.0}
\item{dataA is greater than dataB}
\end{itemize}Statistical testing was carried out as follows: \begin{itemize}
\item{The data is dichotomous and unpaired and so the Fisher test of statistical significance and the Odds Ratio test of effect size are used.}
\item{These results were obtained with 20 samples in each dataset}
\end{itemize}Results:
\begin{itemize}
\item{data is not significant (p value of 0.7511863074565451, threshold of 0.05)}
\end{itemize}Statistical testing was carried out as follows: \begin{itemize}
\item{The data is dichotomous and unpaired and so the Fisher test of statistical significance and the Odds Ratio test of effect size are used.}
\item{These results were obtained with 50 samples in each dataset}
\end{itemize}Results:
\begin{itemize}
\item{data is not significant (p value of 0.9999999999999819, threshold of 0.05)}
\end{itemize}Statistical testing was carried out as follows: \begin{itemize}
\item{The data is dichotomous and unpaired and so the Fisher test of statistical significance and the Odds Ratio test of effect size are used.}
\item{These results were obtained with 50 samples in each dataset}
\end{itemize}Results:
\begin{itemize}
\item{difference is significant (p value of 0.009488575318860788, threshold of 0.05)}
\item{the effect size is 0.22164412070759626}
\item{dataA is lower than dataB}
\end{itemize}Statistical testing was carried out as follows: \begin{itemize}
\item{The data is dichotomous and unpaired and so the Fisher test of statistical significance and the Odds Ratio test of effect size are used.}
\item{These results were obtained with 50 samples in each dataset}
\end{itemize}Results:
\begin{itemize}
\item{difference is significant (p value of 0.008471290461855314, threshold of 0.05)}
\item{the effect size is 0.3182143709715922}
\item{dataA is lower than dataB}
\end{itemize}Statistical testing was carried out as follows: \begin{itemize}
\item{The data is dichotomous and unpaired and so the Fisher test of statistical significance and the Odds Ratio test of effect size are used.}
\item{These results were obtained with 50 samples in each dataset}
\end{itemize}Results:
\begin{itemize}
\item{difference is significant (p value of 5.487013885134591E-10, threshold of 0.05)}
\item{the effect size is 16.849624060150376}
\item{dataA is greater than dataB}
\end{itemize}Statistical testing was carried out as follows: \begin{itemize}
\item{The data is dichotomous and unpaired and so the Fisher test of statistical significance and the Odds Ratio test of effect size are used.}
\item{These results were obtained with 20 runs of metaheuristic dataA and 50 runs of metaheuristic dataB}
\end{itemize}Results:
\begin{itemize}
\item{difference is significant (p value of 4.719176169198642E-6, threshold of 0.05)}
\item{the effect size is 0.0707070707070707}
\item{dataA is lower than dataB}
\end{itemize}

\section{References}
\bibliographystyle{plain}

\begin{thebibliography}{99}
\bibitem{Neumann:2014:EET:2598394.2609850}
Geoffrey Neumann, Jerry Swan, Mark Harman and John A. Clark,
The Executable Experimental Template Pattern for the Systematic Comparison of Metaheuristics,
GECCO Comp '14,
2014
\bibitem{Arcuri2012}
Andrea Arcuri and Lionel Briand,
A hitchhiker's guide to statistical tests for assessing randomized algorithms in software engineering,
Software Testing, Verification and Reliability, 24 (3),
2012
\bibitem{Brunner2000}
Edgar Brunner and Ullrich Munzel, 
The Nonparametric Behrens-Fisher Problem: Asymptotic Theory and aSmall-Sample Approximation, 
Biometrical Journal, 42 (1), 
2000, 
pages 17--25
\bibitem{Gibbons2011}
Jean Dickinson Gibbons and Subhabrata Chakraborti 
Nonparametric statistical inference 
2011, 
Springer\bibitem{Neumann2015}
Geoffrey Neumann, Mark Harman and Simon Poulding 
Transformed Vargha Delaney Effect Size 
Lecture Notes in Computer Science 
SSBSE 
2015, 
pages 318--324
\end{thebibliography}

\end{document}
